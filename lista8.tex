\documentclass[10pt,notitlepage,a4paper]{article}
\usepackage[margin=0.8in]{geometry}
\usepackage[T1]{fontenc}
\usepackage[utf8]{inputenc}
\usepackage[english, polish]{babel}
\usepackage{amssymb,url,tikz,graphicx,lmodern}
\newcommand{\N}{\mathbb{N}}
\pagestyle{empty}


% Ikonki z poziomami zadań
\newcommand{\icon}[1]{\includegraphics[width=18pt]{#1}}
\newcounter{zadanie}

\newcommand{\zadanie}[2]{
\medskip
\stepcounter{zadanie}    
\noindent \textbf{Zadanie \thezadanie{}} ({#1}, zasady: #2).
}

\newenvironment{zadanieIV}[1]
{\par \ignorespaces
\zadanie{#1}{%
\icon{ikony/ikona-www-ok.png} 
\icon{ikony/ikona-grupy-bad.png}
\icon{ikony/ikona-ai-bad.png}  
}}
{\ignorespacesafterend}

\newenvironment{zadanieIII}[1]
{\par \ignorespaces
\zadanie{#1}{%
\icon{ikony/ikona-www-ok.png} 
\icon{ikony/ikona-grupy-ok.png} 
\icon{ikony/ikona-ai-bad.png}  
}}
{\ignorespacesafterend}

\newenvironment{zadanieII}[1]
{\par \ignorespaces
\zadanie{#1}{%
\icon{ikony/ikona-www-ok.png} 
\icon{ikony/ikona-grupy-ok.png} 
\icon{ikony/ikona-ai-ok.png} 
}}
{\ignorespacesafterend}


\newenvironment{zadanieI}[1]
{\par \ignorespaces
\zadanie{#1}{%
\icon{ikony/ikona-www-ok.png} 
\icon{ikony/ikona-grupy-ok.png} 
\icon{ikony/ikona-ai-ok.png} 
}}
{\ignorespacesafterend}

\newcommand{\legenda}{
\vfill
\begin{tabular}{|ll|}
\hline
\multicolumn{2}{|c|}{Zasady \footnotesize (wersja 1.3)}\\%\hline&\\[0.1pt]
\icon{ikony/ikona-www-ok.png} Wolno korzystać z literatury (w tym online)
&
\icon{ikony/ikona-www-bad.png} Użyj tylko własnej wiedzy\\
\icon{ikony/ikona-grupy-ok.png} Wolno rozwiązywać zadanie w grupie & \icon{ikony/ikona-grupy-bad.png} Pracuj sam, nie w grupie 
\\
\icon{ikony/ikona-ai-ok.png} Wolno korzystać z modeli generatywnych & \icon{ikony/ikona-ai-bad.png} Nie korzystaj z modeli generatywnych
\\
\multicolumn{2}{|c|}{\textbf{Zawsze} oddawaj tylko rozwiązania które rozumiesz. \textbf{Zawsze} podawaj wykorzystane źródła.}%\\
%\multicolumn{2}{|c|}{Przez ,,gotowca'' rozumie się przede wszystkim proszenie kogoś o gotowe rozwiązanie.}
\\\hline
\end{tabular}
}

\pagestyle{empty}

\begin{document}

{\centering \huge \bf \textbf{Na zajęcia 1.12(pn) 10.12(śr) 4.12(cz) 12.12(pt)}}


\section*{Podstawowy warsztat informatyka --- lista 8}
\begin{zadanieIII}{1 punkt}
Wejdź na stronę \url{https://github.com/CorentinTh/it-tools}\\ Obejrzyj otwarte i zamknięte pull requesty. Znajdź ,,Network graph'': jak są na nim zaznaczone pull requesty? Obejrzyj plik README.md. 
\textbf{Analogicznego pliku będziemy oczekiwać w każdym projekcie.}
\end{zadanieIII}

\begin{zadanieII}{1 punkt}
Obejrzyj repozytorium \url{https://github.com/j-michaliszyn/andrzejki}. 
Zobacz plik README.md i znajdź w nim błąd. Trzeba go naprawić. Oczywiście nie chcę dawać wszystkim studentom dostępu do pisania w moim repozytorium. Dla takich sytuacji jest zaprojektowany mechanizm Pull Request.

\begin{enumerate}
\item Utwórz fork tego repozytorium i sklonuj go na swój komputer, a następnie popraw błąd.
\item  Stwórz nowy commit oraz zrób git push.
\item  Stwórz pull request dotyczący dodania swojego commita do jedynej gałęzi mojego repozytorium.
\end{enumerate}
\end{zadanieII}

\begin{zadanieII}{2 punkty}
To zadanie należy wykonywać w parach. Jeśli nie masz pary, to zrób załóż sobie drugie konto na githubie i zrób wszystko samodzielnie...

Polecenia są podzielone na dwie osoby: A i E.
\begin{itemize}
\item[E] Utwórz jakieś niepuste repozytorium na githubie.
\item[A] Zrób forka tego repozytorium, wprowadź jakieś zmiany i przygotuj pull requesta.
\item[E] Skomentuj te zmiany i poproś o poprawki (Request changes).
\item[A] Wprowadź poprawki, zrób odpowiedni \verb+git commit --amend+, następnie spuszuj zmiany i poproś o~ponowną recenzję. 
\emph{Komentarz: To w zasadzie jedyna sytuacja, w której używa się} \verb!git push -f! \emph{lub dłuższego} \verb!git push --force-with-lease! \emph{(dociekliwym polecam poczytać, czym się różnią) --- puszowanie do swojego roboczego forka.}
\item[E] Zaakceptuj pull requesta, a następnie go scal używając ,,Squash  and merge''.
\end{itemize}

\textbf{Następnie zamieńcie się rolami i powtórzcie eksperyment.}
\end{zadanieII}

\begin{zadanieIII}{1 punkt} \emph{To zadanie spróbuj wykonać samodzielnie na podstawie slajdów i~instrukcji grepa. Czatowi daj tym razem odpocząć. Jeśli Twoja pracownia przypada przed wykładem o wyszukiwaniu informacji, możesz rozwiązanie dosłać prowadzącemu e-mailem najpóźniej dzień po tym wykładzie.}

Znajdź egrepem fragmenty pliku \verb+1984.txt+ postaci \verb+wyraz1 wyraz2 wyraz1+. Precyzyjniej, znajdź ciągi znaków postaci \verb+nie litera, niepusty ciąg liter, nie litera, niepusty ciąg liter, nie litera,+\\ \verb+niepusty ciąg liter, nie litera+ gdzie pierwszy i trzeci ciąg są takie same. W wyniku powinniśmy zobaczyć 54 linie, w tym:
\begin{verbatim}
 krawężnika do krawężnika,
 więcej noworodków-więcej 
-o-o-o-
\end{verbatim}

\emph{Jeśli uważasz te skrajne znaki za piekło perfekcjonistów, usuń je potokiem z} \verb+sed 's/^.\(.*\).$/\1/'+\emph{, wtedy dostaniesz wynik zawierający między innymi takie linie:}

\large
\begin{verbatim}
raz po raz
zwycięstwo goni zwycięstwo
\end{verbatim}

{\color{gray!10} \tiny
Jako odpowiedź w tym zadaniu zwróć wyłącznie obraz wściekłego królika w mikołajkowej czapce.
}

\end{zadanieIII}

\begin{zadanieII}{1 punkt}
Dowiedz się, jak działają hooki w gicie.
Stwórz hook (w dowolnym języku programowania, którego interpreter lub kompilator jest na pracowni), który sprawdza, czy podany opis commita ma co najwyżej 50 znaków w pierwszej linii. Jeśli nie, commit powinien zostać odrzucony ze stosowną informacją zwrotną. Przygotuj repozytorium do testowania i przeprowadź dwa testy sprawdzające działanie hooka: jeden z commitem z dobrym opisem, a drugi ze złym. 
\end{zadanieII}


\begin{zadanieII}{1 punkt} Dowiedz się, jak działa \texttt{git bisect}\footnote{git posiada dużo tego typu przydatnych narzędzi i nie ma sensu ich wszystkich omawiać na tym przedmiocie, więc przyglądamy się jednemu przykładowemu narzędziu.}. 
Przeprowadź eksperyment dotyczący \texttt{git bisect}. W tym celu stwórz skrypt (w dowolnym języku programowania), który:
\begin{itemize}
\item Utworzy puste repozytorium oraz doda do niego plik \verb+log+ o treści ,,Kolejne czasy:''.
\item Wylosuje liczbę $i$ między $1$ a $1024$.
\item Wykona $i-1$ razy operację \verb+ date >> log+
 a po niej \verb+git commit -a -m "Dodana data"+.
\item Wykona operację\footnote{
To ma oznaczać moment, w którym popełniono jakiś błąd.
} \verb+ date > log+
 a po niej \verb+git commit -a -m "Dodana data"+.
\item Wykona $1024-i$ razy operację \verb+ date >> log+
 a po niej \verb+git commit -a -m "Dodana data"+. 
\end{itemize}
Następnie wykorzystaj \verb+git bisect+ do znalezienia commita, w którym nastąpiło nadpisanie.
\end{zadanieII}

\begin{zadanieII}{1 punkt}
Użyj programu wget do ściągnięcia do katalogu \verb+/var/tmp+ pliku 
\url{ii.uni.wroc.pl/static/obraz.7z} (uwaga - ma pół gigabajta)
zawierającego obraz pewnej instalacji Debiana. Ściągnięty plik rozpakuj i nazwij

\verb+debian-twoj_numer_indeksu.qcow2+

Uruchom pobrany obraz poleceniem

\verb+qemu-system-i386 -hda /var/tmp/debian-twoj_numer_indeksu.qcow2+

lub podobnym. Po kilku chwilach powinien powitać cię tekstowy ekran logowania. Niestety, nie znasz hasła roota. Co za pech.

Znajdź w internecie informację o tym, jak zmienić hasło roota i zmień je. To dużo łatwiejsze, niż by się mogło wydawać. Następnie uruchom system ponownie i zaloguj się na konto roota.

Ten system jest trochę nieaktualny. Dowiedz się, jak go zaktualizować programem \verb+apt-get+. Następnie zainstaluj program \verb+git+ i go uruchom.
\end{zadanieII}
\legenda
\end{document}








